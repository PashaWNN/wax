{\bfseries W\+AX} это простой микройфреймворк, предоставляющий возможности построения пользовательских интерфейсов, направленных, в первую очередь, на использование в среде системного администрирования для изменения настроек сервера или каких-\/либо приложений на нём.

\subsubsection*{Режимы работы}

Приложение, построенное на {\itshape W\+AX} может работать в двух режимах\+: в {\itshape режиме терминала} и в {\itshape режиме сервера}. В первом случае приложение показывает интерфейс прямо в терминале с помощью библиотеки {\itshape curses}, а во втором — строит такой же интерфейс при подключении через веб-\/браузер

\subsubsection*{Пример кода}

Вот пример простой программы, которая складывает(или вычитает) два введённых числа. При запуске без параметров она запустится в терминале, а с флагами {\ttfamily -\/w} и, опционально, {\ttfamily -\/p}, запустит сервер, доступный по {\ttfamily \href{http://localhost}{\tt http\+://localhost}} 
\begin{DoxyCode}
1 from wax.Core import WObject, WForm, bind, bindkey
2 from wax.Components import *
3 from wax.WaxCurses import WaxCurses
4 from wax.WaxServer import WaxServer
5 import argparse as ap
6 KEY\_ESCAPE = 27
7 
8 p = ap.ArgumentParser()                                                  #Parsing command line arguments
9 p.add\_argument("-w", "--web-server", dest='web', action='store\_true',
10                help='Start a web-server')
11 p.add\_argument('-p', '--port', type=int, default=80)
12 a = p.parse\_args()
13 
14 if a.web:                                                    #If '-w' specified
15   w = WaxServer(port=a.port)                                 #Run server on -p port
16 else:                                                        #Else
17   w = WaxCurses()                                            #Run in terminal
18 
19 @bind(w, 'index')                                            #Form to show by default
20 def form\_calc(args):
21   wf = WForm('Calculator')                                   #Title
22   wf.add\_object(WTextEdit('A:', '2', name='a'))              #TextEdit's
23   wf.add\_object(WTextEdit('B:', '4', name='b'))              #Values will be stored in args['variables'] as
       a dict
24   wf.add\_object(WButton("Add", action="add"))                #A button for first action
25   wf.add\_object(WButton("Subtract", action="subtract"))      #And for second
26   return wf
27 
28 @bind(w, 'add')
29 @bind(w, 'subtract')
30 def form\_show\_result(args):                                  #Form to show if action == 'add' or 'subtract'
31   wf = WForm('Result')
32   try:
33     a = int(args['variables']['a'])                          #Restoring vars from dictionary
34     b = int(args['variables']['b'])
35     if args['action'] == 'add':                              #Calculating
36       result = 'Result is ' + str(a+b)
37     elif args['action'] == 'subtract':
38       result = 'Result is ' + str(a-b)
39     wf.add\_object(WLabel(text=result))                       #Output result
40   except ValueError:                                         #Or, if input is not integers, show error
41     wf.add\_object(WLabel('Invalid input provided!'))
42   wf.add\_object(WButton("Return", action="index"))           #Add a button to return
43   return wf
44 
45 @bindkey(w, KEY\_ESCAPE)                                      #Affect only if "w" is WaxCurses
46 def terminate():
47   exit()
48 
49 w.start()
\end{DoxyCode}
 \subsubsection*{Как это выглядит}

\href{https://raw.githubusercontent.com/PashaWNN/wax/master/docs/web.png}{\tt Режим сервера} \href{https://raw.githubusercontent.com/PashaWNN/wax/master/docs/terminal.png}{\tt Режим терминала}

\subsubsection*{Документация}

\href{http://pashawnn.github.io/wax/docs}{\tt Doxygen} {\itshape Позже здесь будет ссылка на туториал на русском языке}

\subsubsection*{Контакты}

\href{http://pashawnn.ru}{\tt Мой сайт}

Вы всегда можете написать мне на почту \href{mailto:pashawnn@pashawnn.ru}{\tt pashawnn@pashawnn.\+ru} или в Telegram  